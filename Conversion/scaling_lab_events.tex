
% Default to the notebook output style

    


% Inherit from the specified cell style.




    
\documentclass[11pt]{article}

    
    \usepackage[T1]{fontenc}
    % Nicer default font (+ math font) than Computer Modern for most use cases
    \usepackage{mathpazo}

    % Basic figure setup, for now with no caption control since it's done
    % automatically by Pandoc (which extracts ![](path) syntax from Markdown).
    \usepackage{graphicx}
    % We will generate all images so they have a width \maxwidth. This means
    % that they will get their normal width if they fit onto the page, but
    % are scaled down if they would overflow the margins.
    \makeatletter
    \def\maxwidth{\ifdim\Gin@nat@width>\linewidth\linewidth
    \else\Gin@nat@width\fi}
    \makeatother
    \let\Oldincludegraphics\includegraphics
    % Set max figure width to be 80% of text width, for now hardcoded.
    \renewcommand{\includegraphics}[1]{\Oldincludegraphics[width=.8\maxwidth]{#1}}
    % Ensure that by default, figures have no caption (until we provide a
    % proper Figure object with a Caption API and a way to capture that
    % in the conversion process - todo).
    \usepackage{caption}
    \DeclareCaptionLabelFormat{nolabel}{}
    \captionsetup{labelformat=nolabel}

    \usepackage{adjustbox} % Used to constrain images to a maximum size 
    \usepackage{xcolor} % Allow colors to be defined
    \usepackage{enumerate} % Needed for markdown enumerations to work
    \usepackage{geometry} % Used to adjust the document margins
    \usepackage{amsmath} % Equations
    \usepackage{amssymb} % Equations
    \usepackage{textcomp} % defines textquotesingle
    % Hack from http://tex.stackexchange.com/a/47451/13684:
    \AtBeginDocument{%
        \def\PYZsq{\textquotesingle}% Upright quotes in Pygmentized code
    }
    \usepackage{upquote} % Upright quotes for verbatim code
    \usepackage{eurosym} % defines \euro
    \usepackage[mathletters]{ucs} % Extended unicode (utf-8) support
    \usepackage[utf8x]{inputenc} % Allow utf-8 characters in the tex document
    \usepackage{fancyvrb} % verbatim replacement that allows latex
    \usepackage{grffile} % extends the file name processing of package graphics 
                         % to support a larger range 
    % The hyperref package gives us a pdf with properly built
    % internal navigation ('pdf bookmarks' for the table of contents,
    % internal cross-reference links, web links for URLs, etc.)
    \usepackage{hyperref}
    \usepackage{longtable} % longtable support required by pandoc >1.10
    \usepackage{booktabs}  % table support for pandoc > 1.12.2
    \usepackage[inline]{enumitem} % IRkernel/repr support (it uses the enumerate* environment)
    \usepackage[normalem]{ulem} % ulem is needed to support strikethroughs (\sout)
                                % normalem makes italics be italics, not underlines
    \usepackage{mathrsfs}
    

    
    
    % Colors for the hyperref package
    \definecolor{urlcolor}{rgb}{0,.145,.698}
    \definecolor{linkcolor}{rgb}{.71,0.21,0.01}
    \definecolor{citecolor}{rgb}{.12,.54,.11}

    % ANSI colors
    \definecolor{ansi-black}{HTML}{3E424D}
    \definecolor{ansi-black-intense}{HTML}{282C36}
    \definecolor{ansi-red}{HTML}{E75C58}
    \definecolor{ansi-red-intense}{HTML}{B22B31}
    \definecolor{ansi-green}{HTML}{00A250}
    \definecolor{ansi-green-intense}{HTML}{007427}
    \definecolor{ansi-yellow}{HTML}{DDB62B}
    \definecolor{ansi-yellow-intense}{HTML}{B27D12}
    \definecolor{ansi-blue}{HTML}{208FFB}
    \definecolor{ansi-blue-intense}{HTML}{0065CA}
    \definecolor{ansi-magenta}{HTML}{D160C4}
    \definecolor{ansi-magenta-intense}{HTML}{A03196}
    \definecolor{ansi-cyan}{HTML}{60C6C8}
    \definecolor{ansi-cyan-intense}{HTML}{258F8F}
    \definecolor{ansi-white}{HTML}{C5C1B4}
    \definecolor{ansi-white-intense}{HTML}{A1A6B2}
    \definecolor{ansi-default-inverse-fg}{HTML}{FFFFFF}
    \definecolor{ansi-default-inverse-bg}{HTML}{000000}

    % commands and environments needed by pandoc snippets
    % extracted from the output of `pandoc -s`
    \providecommand{\tightlist}{%
      \setlength{\itemsep}{0pt}\setlength{\parskip}{0pt}}
    \DefineVerbatimEnvironment{Highlighting}{Verbatim}{commandchars=\\\{\}}
    % Add ',fontsize=\small' for more characters per line
    \newenvironment{Shaded}{}{}
    \newcommand{\KeywordTok}[1]{\textcolor[rgb]{0.00,0.44,0.13}{\textbf{{#1}}}}
    \newcommand{\DataTypeTok}[1]{\textcolor[rgb]{0.56,0.13,0.00}{{#1}}}
    \newcommand{\DecValTok}[1]{\textcolor[rgb]{0.25,0.63,0.44}{{#1}}}
    \newcommand{\BaseNTok}[1]{\textcolor[rgb]{0.25,0.63,0.44}{{#1}}}
    \newcommand{\FloatTok}[1]{\textcolor[rgb]{0.25,0.63,0.44}{{#1}}}
    \newcommand{\CharTok}[1]{\textcolor[rgb]{0.25,0.44,0.63}{{#1}}}
    \newcommand{\StringTok}[1]{\textcolor[rgb]{0.25,0.44,0.63}{{#1}}}
    \newcommand{\CommentTok}[1]{\textcolor[rgb]{0.38,0.63,0.69}{\textit{{#1}}}}
    \newcommand{\OtherTok}[1]{\textcolor[rgb]{0.00,0.44,0.13}{{#1}}}
    \newcommand{\AlertTok}[1]{\textcolor[rgb]{1.00,0.00,0.00}{\textbf{{#1}}}}
    \newcommand{\FunctionTok}[1]{\textcolor[rgb]{0.02,0.16,0.49}{{#1}}}
    \newcommand{\RegionMarkerTok}[1]{{#1}}
    \newcommand{\ErrorTok}[1]{\textcolor[rgb]{1.00,0.00,0.00}{\textbf{{#1}}}}
    \newcommand{\NormalTok}[1]{{#1}}
    
    % Additional commands for more recent versions of Pandoc
    \newcommand{\ConstantTok}[1]{\textcolor[rgb]{0.53,0.00,0.00}{{#1}}}
    \newcommand{\SpecialCharTok}[1]{\textcolor[rgb]{0.25,0.44,0.63}{{#1}}}
    \newcommand{\VerbatimStringTok}[1]{\textcolor[rgb]{0.25,0.44,0.63}{{#1}}}
    \newcommand{\SpecialStringTok}[1]{\textcolor[rgb]{0.73,0.40,0.53}{{#1}}}
    \newcommand{\ImportTok}[1]{{#1}}
    \newcommand{\DocumentationTok}[1]{\textcolor[rgb]{0.73,0.13,0.13}{\textit{{#1}}}}
    \newcommand{\AnnotationTok}[1]{\textcolor[rgb]{0.38,0.63,0.69}{\textbf{\textit{{#1}}}}}
    \newcommand{\CommentVarTok}[1]{\textcolor[rgb]{0.38,0.63,0.69}{\textbf{\textit{{#1}}}}}
    \newcommand{\VariableTok}[1]{\textcolor[rgb]{0.10,0.09,0.49}{{#1}}}
    \newcommand{\ControlFlowTok}[1]{\textcolor[rgb]{0.00,0.44,0.13}{\textbf{{#1}}}}
    \newcommand{\OperatorTok}[1]{\textcolor[rgb]{0.40,0.40,0.40}{{#1}}}
    \newcommand{\BuiltInTok}[1]{{#1}}
    \newcommand{\ExtensionTok}[1]{{#1}}
    \newcommand{\PreprocessorTok}[1]{\textcolor[rgb]{0.74,0.48,0.00}{{#1}}}
    \newcommand{\AttributeTok}[1]{\textcolor[rgb]{0.49,0.56,0.16}{{#1}}}
    \newcommand{\InformationTok}[1]{\textcolor[rgb]{0.38,0.63,0.69}{\textbf{\textit{{#1}}}}}
    \newcommand{\WarningTok}[1]{\textcolor[rgb]{0.38,0.63,0.69}{\textbf{\textit{{#1}}}}}
    
    
    % Define a nice break command that doesn't care if a line doesn't already
    % exist.
    \def\br{\hspace*{\fill} \\* }
    % Math Jax compatability definitions
    \def\gt{>}
    \def\lt{<}
    \let\Oldtex\TeX
    \let\Oldlatex\LaTeX
    \renewcommand{\TeX}{\textrm{\Oldtex}}
    \renewcommand{\LaTeX}{\textrm{\Oldlatex}}
    % Document parameters
    % Document title
    \title{scaling\_lab\_events}
    
    
    

    % Pygments definitions
    
\makeatletter
\def\PY@reset{\let\PY@it=\relax \let\PY@bf=\relax%
    \let\PY@ul=\relax \let\PY@tc=\relax%
    \let\PY@bc=\relax \let\PY@ff=\relax}
\def\PY@tok#1{\csname PY@tok@#1\endcsname}
\def\PY@toks#1+{\ifx\relax#1\empty\else%
    \PY@tok{#1}\expandafter\PY@toks\fi}
\def\PY@do#1{\PY@bc{\PY@tc{\PY@ul{%
    \PY@it{\PY@bf{\PY@ff{#1}}}}}}}
\def\PY#1#2{\PY@reset\PY@toks#1+\relax+\PY@do{#2}}

\expandafter\def\csname PY@tok@w\endcsname{\def\PY@tc##1{\textcolor[rgb]{0.73,0.73,0.73}{##1}}}
\expandafter\def\csname PY@tok@c\endcsname{\let\PY@it=\textit\def\PY@tc##1{\textcolor[rgb]{0.25,0.50,0.50}{##1}}}
\expandafter\def\csname PY@tok@cp\endcsname{\def\PY@tc##1{\textcolor[rgb]{0.74,0.48,0.00}{##1}}}
\expandafter\def\csname PY@tok@k\endcsname{\let\PY@bf=\textbf\def\PY@tc##1{\textcolor[rgb]{0.00,0.50,0.00}{##1}}}
\expandafter\def\csname PY@tok@kp\endcsname{\def\PY@tc##1{\textcolor[rgb]{0.00,0.50,0.00}{##1}}}
\expandafter\def\csname PY@tok@kt\endcsname{\def\PY@tc##1{\textcolor[rgb]{0.69,0.00,0.25}{##1}}}
\expandafter\def\csname PY@tok@o\endcsname{\def\PY@tc##1{\textcolor[rgb]{0.40,0.40,0.40}{##1}}}
\expandafter\def\csname PY@tok@ow\endcsname{\let\PY@bf=\textbf\def\PY@tc##1{\textcolor[rgb]{0.67,0.13,1.00}{##1}}}
\expandafter\def\csname PY@tok@nb\endcsname{\def\PY@tc##1{\textcolor[rgb]{0.00,0.50,0.00}{##1}}}
\expandafter\def\csname PY@tok@nf\endcsname{\def\PY@tc##1{\textcolor[rgb]{0.00,0.00,1.00}{##1}}}
\expandafter\def\csname PY@tok@nc\endcsname{\let\PY@bf=\textbf\def\PY@tc##1{\textcolor[rgb]{0.00,0.00,1.00}{##1}}}
\expandafter\def\csname PY@tok@nn\endcsname{\let\PY@bf=\textbf\def\PY@tc##1{\textcolor[rgb]{0.00,0.00,1.00}{##1}}}
\expandafter\def\csname PY@tok@ne\endcsname{\let\PY@bf=\textbf\def\PY@tc##1{\textcolor[rgb]{0.82,0.25,0.23}{##1}}}
\expandafter\def\csname PY@tok@nv\endcsname{\def\PY@tc##1{\textcolor[rgb]{0.10,0.09,0.49}{##1}}}
\expandafter\def\csname PY@tok@no\endcsname{\def\PY@tc##1{\textcolor[rgb]{0.53,0.00,0.00}{##1}}}
\expandafter\def\csname PY@tok@nl\endcsname{\def\PY@tc##1{\textcolor[rgb]{0.63,0.63,0.00}{##1}}}
\expandafter\def\csname PY@tok@ni\endcsname{\let\PY@bf=\textbf\def\PY@tc##1{\textcolor[rgb]{0.60,0.60,0.60}{##1}}}
\expandafter\def\csname PY@tok@na\endcsname{\def\PY@tc##1{\textcolor[rgb]{0.49,0.56,0.16}{##1}}}
\expandafter\def\csname PY@tok@nt\endcsname{\let\PY@bf=\textbf\def\PY@tc##1{\textcolor[rgb]{0.00,0.50,0.00}{##1}}}
\expandafter\def\csname PY@tok@nd\endcsname{\def\PY@tc##1{\textcolor[rgb]{0.67,0.13,1.00}{##1}}}
\expandafter\def\csname PY@tok@s\endcsname{\def\PY@tc##1{\textcolor[rgb]{0.73,0.13,0.13}{##1}}}
\expandafter\def\csname PY@tok@sd\endcsname{\let\PY@it=\textit\def\PY@tc##1{\textcolor[rgb]{0.73,0.13,0.13}{##1}}}
\expandafter\def\csname PY@tok@si\endcsname{\let\PY@bf=\textbf\def\PY@tc##1{\textcolor[rgb]{0.73,0.40,0.53}{##1}}}
\expandafter\def\csname PY@tok@se\endcsname{\let\PY@bf=\textbf\def\PY@tc##1{\textcolor[rgb]{0.73,0.40,0.13}{##1}}}
\expandafter\def\csname PY@tok@sr\endcsname{\def\PY@tc##1{\textcolor[rgb]{0.73,0.40,0.53}{##1}}}
\expandafter\def\csname PY@tok@ss\endcsname{\def\PY@tc##1{\textcolor[rgb]{0.10,0.09,0.49}{##1}}}
\expandafter\def\csname PY@tok@sx\endcsname{\def\PY@tc##1{\textcolor[rgb]{0.00,0.50,0.00}{##1}}}
\expandafter\def\csname PY@tok@m\endcsname{\def\PY@tc##1{\textcolor[rgb]{0.40,0.40,0.40}{##1}}}
\expandafter\def\csname PY@tok@gh\endcsname{\let\PY@bf=\textbf\def\PY@tc##1{\textcolor[rgb]{0.00,0.00,0.50}{##1}}}
\expandafter\def\csname PY@tok@gu\endcsname{\let\PY@bf=\textbf\def\PY@tc##1{\textcolor[rgb]{0.50,0.00,0.50}{##1}}}
\expandafter\def\csname PY@tok@gd\endcsname{\def\PY@tc##1{\textcolor[rgb]{0.63,0.00,0.00}{##1}}}
\expandafter\def\csname PY@tok@gi\endcsname{\def\PY@tc##1{\textcolor[rgb]{0.00,0.63,0.00}{##1}}}
\expandafter\def\csname PY@tok@gr\endcsname{\def\PY@tc##1{\textcolor[rgb]{1.00,0.00,0.00}{##1}}}
\expandafter\def\csname PY@tok@ge\endcsname{\let\PY@it=\textit}
\expandafter\def\csname PY@tok@gs\endcsname{\let\PY@bf=\textbf}
\expandafter\def\csname PY@tok@gp\endcsname{\let\PY@bf=\textbf\def\PY@tc##1{\textcolor[rgb]{0.00,0.00,0.50}{##1}}}
\expandafter\def\csname PY@tok@go\endcsname{\def\PY@tc##1{\textcolor[rgb]{0.53,0.53,0.53}{##1}}}
\expandafter\def\csname PY@tok@gt\endcsname{\def\PY@tc##1{\textcolor[rgb]{0.00,0.27,0.87}{##1}}}
\expandafter\def\csname PY@tok@err\endcsname{\def\PY@bc##1{\setlength{\fboxsep}{0pt}\fcolorbox[rgb]{1.00,0.00,0.00}{1,1,1}{\strut ##1}}}
\expandafter\def\csname PY@tok@kc\endcsname{\let\PY@bf=\textbf\def\PY@tc##1{\textcolor[rgb]{0.00,0.50,0.00}{##1}}}
\expandafter\def\csname PY@tok@kd\endcsname{\let\PY@bf=\textbf\def\PY@tc##1{\textcolor[rgb]{0.00,0.50,0.00}{##1}}}
\expandafter\def\csname PY@tok@kn\endcsname{\let\PY@bf=\textbf\def\PY@tc##1{\textcolor[rgb]{0.00,0.50,0.00}{##1}}}
\expandafter\def\csname PY@tok@kr\endcsname{\let\PY@bf=\textbf\def\PY@tc##1{\textcolor[rgb]{0.00,0.50,0.00}{##1}}}
\expandafter\def\csname PY@tok@bp\endcsname{\def\PY@tc##1{\textcolor[rgb]{0.00,0.50,0.00}{##1}}}
\expandafter\def\csname PY@tok@fm\endcsname{\def\PY@tc##1{\textcolor[rgb]{0.00,0.00,1.00}{##1}}}
\expandafter\def\csname PY@tok@vc\endcsname{\def\PY@tc##1{\textcolor[rgb]{0.10,0.09,0.49}{##1}}}
\expandafter\def\csname PY@tok@vg\endcsname{\def\PY@tc##1{\textcolor[rgb]{0.10,0.09,0.49}{##1}}}
\expandafter\def\csname PY@tok@vi\endcsname{\def\PY@tc##1{\textcolor[rgb]{0.10,0.09,0.49}{##1}}}
\expandafter\def\csname PY@tok@vm\endcsname{\def\PY@tc##1{\textcolor[rgb]{0.10,0.09,0.49}{##1}}}
\expandafter\def\csname PY@tok@sa\endcsname{\def\PY@tc##1{\textcolor[rgb]{0.73,0.13,0.13}{##1}}}
\expandafter\def\csname PY@tok@sb\endcsname{\def\PY@tc##1{\textcolor[rgb]{0.73,0.13,0.13}{##1}}}
\expandafter\def\csname PY@tok@sc\endcsname{\def\PY@tc##1{\textcolor[rgb]{0.73,0.13,0.13}{##1}}}
\expandafter\def\csname PY@tok@dl\endcsname{\def\PY@tc##1{\textcolor[rgb]{0.73,0.13,0.13}{##1}}}
\expandafter\def\csname PY@tok@s2\endcsname{\def\PY@tc##1{\textcolor[rgb]{0.73,0.13,0.13}{##1}}}
\expandafter\def\csname PY@tok@sh\endcsname{\def\PY@tc##1{\textcolor[rgb]{0.73,0.13,0.13}{##1}}}
\expandafter\def\csname PY@tok@s1\endcsname{\def\PY@tc##1{\textcolor[rgb]{0.73,0.13,0.13}{##1}}}
\expandafter\def\csname PY@tok@mb\endcsname{\def\PY@tc##1{\textcolor[rgb]{0.40,0.40,0.40}{##1}}}
\expandafter\def\csname PY@tok@mf\endcsname{\def\PY@tc##1{\textcolor[rgb]{0.40,0.40,0.40}{##1}}}
\expandafter\def\csname PY@tok@mh\endcsname{\def\PY@tc##1{\textcolor[rgb]{0.40,0.40,0.40}{##1}}}
\expandafter\def\csname PY@tok@mi\endcsname{\def\PY@tc##1{\textcolor[rgb]{0.40,0.40,0.40}{##1}}}
\expandafter\def\csname PY@tok@il\endcsname{\def\PY@tc##1{\textcolor[rgb]{0.40,0.40,0.40}{##1}}}
\expandafter\def\csname PY@tok@mo\endcsname{\def\PY@tc##1{\textcolor[rgb]{0.40,0.40,0.40}{##1}}}
\expandafter\def\csname PY@tok@ch\endcsname{\let\PY@it=\textit\def\PY@tc##1{\textcolor[rgb]{0.25,0.50,0.50}{##1}}}
\expandafter\def\csname PY@tok@cm\endcsname{\let\PY@it=\textit\def\PY@tc##1{\textcolor[rgb]{0.25,0.50,0.50}{##1}}}
\expandafter\def\csname PY@tok@cpf\endcsname{\let\PY@it=\textit\def\PY@tc##1{\textcolor[rgb]{0.25,0.50,0.50}{##1}}}
\expandafter\def\csname PY@tok@c1\endcsname{\let\PY@it=\textit\def\PY@tc##1{\textcolor[rgb]{0.25,0.50,0.50}{##1}}}
\expandafter\def\csname PY@tok@cs\endcsname{\let\PY@it=\textit\def\PY@tc##1{\textcolor[rgb]{0.25,0.50,0.50}{##1}}}

\def\PYZbs{\char`\\}
\def\PYZus{\char`\_}
\def\PYZob{\char`\{}
\def\PYZcb{\char`\}}
\def\PYZca{\char`\^}
\def\PYZam{\char`\&}
\def\PYZlt{\char`\<}
\def\PYZgt{\char`\>}
\def\PYZsh{\char`\#}
\def\PYZpc{\char`\%}
\def\PYZdl{\char`\$}
\def\PYZhy{\char`\-}
\def\PYZsq{\char`\'}
\def\PYZdq{\char`\"}
\def\PYZti{\char`\~}
% for compatibility with earlier versions
\def\PYZat{@}
\def\PYZlb{[}
\def\PYZrb{]}
\makeatother


    % Exact colors from NB
    \definecolor{incolor}{rgb}{0.0, 0.0, 0.5}
    \definecolor{outcolor}{rgb}{0.545, 0.0, 0.0}



    
    % Prevent overflowing lines due to hard-to-break entities
    \sloppy 
    % Setup hyperref package
    \hypersetup{
      breaklinks=true,  % so long urls are correctly broken across lines
      colorlinks=true,
      urlcolor=urlcolor,
      linkcolor=linkcolor,
      citecolor=citecolor,
      }
    % Slightly bigger margins than the latex defaults
    
    \geometry{verbose,tmargin=1in,bmargin=1in,lmargin=1in,rmargin=1in}
    
    

    \begin{document}
    
    
    \maketitle
    
    

    
    \begin{Verbatim}[commandchars=\\\{\},fontsize=\scriptsize]
{\color{incolor}In [{\color{incolor}1}]:} \PY{c+c1}{\PYZsh{} Stefan Nielsen 2018}
        \PY{c+c1}{\PYZsh{}\PYZsh{} the inline option is necessary for Latex export of figures:}
        \PY{o}{\PYZpc{}}\PY{k}{matplotlib} inline
        \PY{k+kn}{import} \PY{n+nn}{os}
        \PY{k+kn}{import} \PY{n+nn}{numpy} \PY{k}{as} \PY{n+nn}{np}
        \PY{k+kn}{import} \PY{n+nn}{matplotlib}
        \PY{k+kn}{import} \PY{n+nn}{matplotlib}\PY{n+nn}{.}\PY{n+nn}{pyplot} \PY{k}{as} \PY{n+nn}{plt}
        \PY{k+kn}{from} \PY{n+nn}{pylab} \PY{k}{import} \PY{n}{plot}\PY{p}{,} \PY{n}{xlabel}\PY{p}{,} \PY{n}{ylabel}
        \PY{k+kn}{import} \PY{n+nn}{pandas} \PY{k}{as} \PY{n+nn}{pd}
        \PY{k+kn}{from} \PY{n+nn}{xlrd} \PY{k}{import} \PY{n}{open\PYZus{}workbook}
        \PY{c+c1}{\PYZsh{}\PYZsh{} This sets PDF format for export to LaTeX,}
        \PY{c+c1}{\PYZsh{}\PYZsh{} while allowing inline SVG in the notebook:}
        \PY{k+kn}{from} \PY{n+nn}{IPython}\PY{n+nn}{.}\PY{n+nn}{display} \PY{k}{import} \PY{n}{set\PYZus{}matplotlib\PYZus{}formats}
        \PY{n}{set\PYZus{}matplotlib\PYZus{}formats}\PY{p}{(}\PY{l+s+s1}{\PYZsq{}}\PY{l+s+s1}{svg}\PY{l+s+s1}{\PYZsq{}}\PY{p}{,} \PY{l+s+s1}{\PYZsq{}}\PY{l+s+s1}{pdf}\PY{l+s+s1}{\PYZsq{}}\PY{p}{)}
\end{Verbatim}

    \begin{Verbatim}[commandchars=\\\{\},fontsize=\scriptsize]
{\color{incolor}In [{\color{incolor}2}]:} \PY{c+c1}{\PYZsh{}\PYZsh{} Alternative options:}
        \PY{c+c1}{\PYZsh{}\PYZpc{}matplotlib notebook}
        \PY{c+c1}{\PYZsh{}\PYZpc{}config InlineBackend.figure\PYZus{}format = \PYZsq{}pdf\PYZsq{}}
        \PY{c+c1}{\PYZsh{}\PYZpc{}config InlineBackend.figure\PYZus{}format = \PYZsq{}png\PYZsq{}}
        \PY{c+c1}{\PYZsh{}\PYZpc{}config InlineBackend.figure\PYZus{}format = \PYZsq{}svg\PYZsq{}}
        \PY{c+c1}{\PYZsh{}from matplotlib import animation, rc, interactive}
        \PY{c+c1}{\PYZsh{}import matplotlib.ticker as ticker}
        \PY{c+c1}{\PYZsh{}matplotlib.interactive(True)}
        \PY{c+c1}{\PYZsh{}from pylab import *}
        \PY{c+c1}{\PYZsh{}from scipy import arange}
        \PY{c+c1}{\PYZsh{}from IPython import display}
        \PY{c+c1}{\PYZsh{}plt.rcParams.update(\PYZob{}\PYZsq{}figure.figsize\PYZsq{}: (10,7)\PYZcb{})}
        \PY{c+c1}{\PYZsh{} LaTeX support, with pslatex package :}
        \PY{c+c1}{\PYZsh{}plt.rc(\PYZsq{}text\PYZsq{}, usetex=True);plt.rc(\PYZsq{}font\PYZsq{}, family=\PYZsq{}serif\PYZsq{})}
        \PY{c+c1}{\PYZsh{}matplotlib.rcParams[\PYZsq{}text.latex.preamble\PYZsq{}] = [r\PYZsq{}\PYZbs{}usepackage\PYZob{}amsmath\PYZcb{}\PYZsq{},r\PYZsq{}\PYZbs{}usepackage\PYZob{}pslatex\PYZcb{}\PYZsq{}]}
\end{Verbatim}

    \section{Parameters of rock samples:}\label{parameters-of-rock-samples}

\[\begin{split}
\mu'=24.3\ \textrm{ GPa}\\
\lambda=39.1\ \textrm{ GPa}\\
\rho=2700 \textrm{ kg m}^{-3}\\
V_p= 5699\textrm{ m/s}\\ 
V_s= 3000\textrm{ m/s}
\end{split}\]

    \subsection{import xlsx file, show contents, use
contents}\label{import-xlsx-file-show-contents-use-contents}

    \begin{Verbatim}[commandchars=\\\{\},fontsize=\scriptsize]
{\color{incolor}In [{\color{incolor}3}]:} \PY{n}{mu}\PY{o}{=}\PY{l+m+mf}{24.3e9}
        \PY{n}{df}\PY{o}{=}\PY{n}{pd}\PY{o}{.}\PY{n}{read\PYZus{}excel}\PY{p}{(}\PY{l+s+s2}{\PYZdq{}}\PY{l+s+s2}{event\PYZus{}params.xlsx}\PY{l+s+s2}{\PYZdq{}}\PY{p}{)}
        \PY{n}{df}
\end{Verbatim}

\begin{Verbatim}[commandchars=\\\{\}]
{\color{outcolor}Out[{\color{outcolor}3}]:}      Event  t0/2e-7  t1/2e-7    tw  rise time (s)        tc         Sn  mu0  \textbackslash{}
        0  157\_28c     1927     3000  2243       0.000215  0.000063   75000000  0.5   
        1  157\_68e     2003     3038  2277       0.000207  0.000055   75000000  0.5   
        2  159\_184     1984     2725  2539       0.000148  0.000111   80000000  0.5   
        3  159\_237     1984     2772  2298       0.000158  0.000063  100000000  0.5   
        4  159\_240     2000     2777  2589       0.000155  0.000118  100000000  0.5   
        5   160\_27     2000     3981  2329       0.000396  0.000066   58000000  0.5   
        6   160\_79     1981     4191  2352       0.000442  0.000074   73000000  0.5   
        7  160\_124     1955     4230  2390       0.000455  0.000087   75000000  0.5   
        8  160\_130     1867     4300  2398       0.000487  0.000106   76000000  0.5   
        
              muR     ∆mu        ∆tau    Vr    Vmax         U        Dc    Dc ida  
        0  0.4354  0.0646   4845000.0  1100  0.1856  0.000013  0.000005  0.000002  
        1  0.4284  0.0716   5370000.0  1000  0.1417  0.000009  0.000004  0.000002  
        2  0.3557  0.1443  11544000.0  2500  0.1792  0.000012  0.000011  0.000023  
        3  0.3055  0.1945  19450000.0  2190  0.1467  0.000013  0.000006  0.000019  
        4  0.2902  0.2098  20980000.0  2560  0.1632  0.000014  0.000013  0.000045  
        5  0.2253  0.2747  15932600.0  2166  0.3548  0.000083  0.000015  0.000016  
        6  0.1703  0.3297  24068100.0  1460  0.3455  0.000085  0.000016  0.000018  
        7  0.1018  0.3982  29865000.0  1797  0.5300  0.000129  0.000027  0.000033  
        8  0.0968  0.4032  30643200.0  1800  0.5199  0.000127  0.000017  0.000041  
\end{Verbatim}
            
    \begin{Verbatim}[commandchars=\\\{\},fontsize=\scriptsize]
{\color{incolor}In [{\color{incolor}4}]:} \PY{n}{Vmean}\PY{o}{=}\PY{n}{df}\PY{p}{[}\PY{l+s+s2}{\PYZdq{}}\PY{l+s+s2}{U}\PY{l+s+s2}{\PYZdq{}}\PY{p}{]}\PY{o}{/}\PY{n}{df}\PY{p}{[}\PY{l+s+s1}{\PYZsq{}}\PY{l+s+s1}{rise time (s)}\PY{l+s+s1}{\PYZsq{}}\PY{p}{]}
        \PY{n}{nor}\PY{o}{=}\PY{n}{df}\PY{p}{[}\PY{l+s+s1}{\PYZsq{}}\PY{l+s+s1}{Sn}\PY{l+s+s1}{\PYZsq{}}\PY{p}{]}\PY{o}{/}\PY{n}{mu}
\end{Verbatim}

    \section{Make graphics using xlsx
contents:}\label{make-graphics-using-xlsx-contents}

    \begin{Verbatim}[commandchars=\\\{\},fontsize=\scriptsize]
{\color{incolor}In [{\color{incolor}5}]:} \PY{c+c1}{\PYZsh{}\PYZsh{} This sets the dpi resolution for screen and png files:}
        \PY{c+c1}{\PYZsh{}plt.rcParams[\PYZsq{}figure.dpi\PYZsq{}] = 120;plt.figure(dpi=120);}
        \PY{c+c1}{\PYZsh{} for some reason, it does not work if declared in the initial cell of notebook.}
        \PY{c+c1}{\PYZsh{} note that PDF format still takes precedence during export to LaTeX||}
\end{Verbatim}

    \begin{Verbatim}[commandchars=\\\{\},fontsize=\scriptsize]
{\color{incolor}In [{\color{incolor}6}]:} \PY{n}{fig1}\PY{p}{,} \PY{n}{ax1} \PY{o}{=} \PY{n}{plt}\PY{o}{.}\PY{n}{subplots}\PY{p}{(}\PY{p}{)}
        \PY{n}{xa}\PY{o}{=}\PY{l+s+s2}{\PYZdq{}}\PY{l+s+s2}{\PYZdl{}V\PYZus{}}\PY{l+s+s2}{\PYZbs{}}\PY{l+s+s2}{mathrm}\PY{l+s+si}{\PYZob{}max\PYZcb{}}\PY{l+s+s2}{\PYZbs{}}\PY{l+s+s2}{ }\PY{l+s+s2}{\PYZbs{}}\PY{l+s+s2}{mathrm}\PY{l+s+s2}{\PYZob{}}\PY{l+s+s2}{  (m/s)\PYZcb{}\PYZdl{}}\PY{l+s+s2}{\PYZdq{}}\PY{p}{;}\PY{n}{ya}\PY{o}{=}\PY{l+s+s1}{\PYZsq{}}\PY{l+s+s1}{\PYZdl{}}\PY{l+s+s1}{\PYZbs{}}\PY{l+s+s1}{Delta }\PY{l+s+s1}{\PYZbs{}}\PY{l+s+s1}{mu\PYZdl{}}\PY{l+s+s1}{\PYZsq{}}
        \PY{n}{ax1}\PY{o}{.}\PY{n}{plot}\PY{p}{(}\PY{n}{df}\PY{p}{[}\PY{l+s+s2}{\PYZdq{}}\PY{l+s+s2}{Vmax}\PY{l+s+s2}{\PYZdq{}}\PY{p}{]}\PY{p}{,}\PY{p}{(}\PY{n}{df}\PY{p}{[}\PY{l+s+s2}{\PYZdq{}}\PY{l+s+s2}{mu0}\PY{l+s+s2}{\PYZdq{}}\PY{p}{]}\PY{o}{\PYZhy{}}\PY{n}{df}\PY{p}{[}\PY{l+s+s2}{\PYZdq{}}\PY{l+s+s2}{muR}\PY{l+s+s2}{\PYZdq{}}\PY{p}{]}\PY{p}{)}\PY{p}{,}\PY{l+s+s1}{\PYZsq{}}\PY{l+s+s1}{ro}\PY{l+s+s1}{\PYZsq{}}\PY{p}{)}
        \PY{n}{xlabel}\PY{p}{(}\PY{n}{xa}\PY{p}{)}\PY{p}{;}\PY{n}{ylabel}\PY{p}{(}\PY{n}{ya}\PY{p}{)}
        \PY{n}{ax1}\PY{o}{.}\PY{n}{set\PYZus{}xlim}\PY{p}{(}\PY{n}{left}\PY{o}{=}\PY{l+m+mi}{0}\PY{p}{)}\PY{p}{;}\PY{n}{ax1}\PY{o}{.}\PY{n}{set\PYZus{}ylim}\PY{p}{(}\PY{n}{bottom}\PY{o}{=}\PY{l+m+mi}{0}\PY{p}{)}
        \PY{n}{ax1}\PY{o}{.}\PY{n}{set\PYZus{}xlim}\PY{p}{(}\PY{n}{right}\PY{o}{=}\PY{l+m+mi}{2}\PY{p}{)}\PY{p}{;}\PY{n}{ax1}\PY{o}{.}\PY{n}{set\PYZus{}ylim}\PY{p}{(}\PY{n}{top}\PY{o}{=}\PY{o}{.}\PY{l+m+mi}{55}\PY{p}{)}
        \PY{n}{x}\PY{o}{=}\PY{n}{np}\PY{o}{.}\PY{n}{linspace}\PY{p}{(}\PY{l+m+mf}{0.01}\PY{p}{,}\PY{l+m+mi}{2}\PY{p}{,}\PY{l+m+mi}{100}\PY{p}{)}
        \PY{n}{y}\PY{o}{=}\PY{l+m+mf}{0.5}\PY{o}{*}\PY{p}{(}\PY{l+m+mi}{1}\PY{o}{\PYZhy{}}\PY{o}{.}\PY{l+m+mi}{12}\PY{o}{/}\PY{n}{x}\PY{p}{)}
        \PY{n}{ax1}\PY{o}{.}\PY{n}{plot}\PY{p}{(}\PY{n}{x}\PY{p}{,}\PY{n}{y}\PY{p}{)}\PY{p}{;}
        \PY{n}{y}\PY{o}{=}\PY{o}{.}\PY{l+m+mi}{5}\PY{o}{+}\PY{l+m+mi}{0}\PY{o}{*}\PY{n}{x}
        \PY{n}{ax1}\PY{o}{.}\PY{n}{plot}\PY{p}{(}\PY{n}{x}\PY{p}{,}\PY{n}{y}\PY{p}{)}\PY{p}{;}
\end{Verbatim}

    \begin{center}
    \adjustimage{max size={0.9\linewidth}{0.9\paperheight}}{scaling_lab_events_files/scaling_lab_events_8_0.pdf}
    \end{center}
    { \hspace*{\fill} \\}
    
    Fig. 1. Stress drop \(\Delta\mu=\mu_0-\mu_{dyn}\) as a function of
maximum slip rate \(V_\mathrm{max}\) in different experimental
microearthquakes (red dots). The theoretical fit (blue curve) uses
\(\Delta\mu=\mu_0(1-V_w/V)\), which results from
\(\mu_{dyn}=\mu_0\ V_0/V\), a high-velocity (\(V\gg V_w\)) approximation
of the flash weakening law. Here \(V_w=0.12\textrm{ m/s}\) and
\(\mu_0=0.5\) (orange line, or total stress drop, reached asymtotically
for \(V\rightarrow\infty\)).

    \begin{Verbatim}[commandchars=\\\{\},fontsize=\scriptsize]
{\color{incolor}In [{\color{incolor}7}]:} \PY{n}{fig4}\PY{p}{,} \PY{n}{ax1} \PY{o}{=} \PY{n}{plt}\PY{o}{.}\PY{n}{subplots}\PY{p}{(}\PY{p}{)}
        \PY{n}{xa}\PY{o}{=}\PY{l+s+s1}{\PYZsq{}}\PY{l+s+s1}{rise time (s)}\PY{l+s+s1}{\PYZsq{}}\PY{p}{;}\PY{n}{ya}\PY{o}{=}\PY{l+s+s1}{\PYZsq{}}\PY{l+s+s1}{U}\PY{l+s+s1}{\PYZsq{}}
        \PY{n}{ax1}\PY{o}{.}\PY{n}{plot}\PY{p}{(}\PY{n}{df}\PY{p}{[}\PY{p}{[}\PY{n}{xa}\PY{p}{]}\PY{p}{]}\PY{p}{,}\PY{n}{df}\PY{p}{[}\PY{p}{[}\PY{n}{ya}\PY{p}{]}\PY{p}{]}\PY{p}{,}\PY{l+s+s1}{\PYZsq{}}\PY{l+s+s1}{ro}\PY{l+s+s1}{\PYZsq{}}\PY{p}{)}\PY{p}{;}
        \PY{n}{xlabel}\PY{p}{(}\PY{n}{xa}\PY{p}{)}\PY{p}{;}\PY{n}{ylabel}\PY{p}{(}\PY{n}{ya}\PY{p}{)}
        \PY{n}{ax1}\PY{o}{.}\PY{n}{set\PYZus{}xlim}\PY{p}{(}\PY{n}{left}\PY{o}{=}\PY{l+m+mi}{0}\PY{p}{)}\PY{p}{;}\PY{n}{ax1}\PY{o}{.}\PY{n}{set\PYZus{}ylim}\PY{p}{(}\PY{n}{bottom}\PY{o}{=}\PY{l+m+mi}{0}\PY{p}{)}
        \PY{c+c1}{\PYZsh{}ax1.set\PYZus{}xlim(left=0,right=1e\PYZhy{}3);ax1.set\PYZus{}ylim(bottom=0,top=.3e\PYZhy{}3)}
        \PY{n}{x}\PY{o}{=}\PY{n}{np}\PY{o}{.}\PY{n}{linspace}\PY{p}{(}\PY{l+m+mi}{0}\PY{p}{,}\PY{l+m+mf}{2e\PYZhy{}3}\PY{p}{,}\PY{l+m+mi}{100}\PY{p}{)}
        \PY{n}{y}\PY{o}{=}\PY{l+m+mf}{5e2}\PY{o}{*}\PY{n}{x}\PY{o}{*}\PY{o}{*}\PY{l+m+mi}{2}
        \PY{n}{vvmax}\PY{o}{=}\PY{o}{.}\PY{l+m+mi}{12}\PY{o}{+}\PY{p}{(}\PY{n}{x}\PY{o}{/}\PY{l+m+mf}{8e\PYZhy{}4}\PY{p}{)}\PY{o}{*}\PY{o}{*}\PY{l+m+mi}{2}
        \PY{n}{y}\PY{o}{=}\PY{o}{.}\PY{l+m+mi}{36}\PY{o}{*}\PY{n}{x}\PY{o}{*}\PY{p}{(}\PY{l+m+mi}{1}\PY{o}{\PYZhy{}}\PY{l+m+mf}{0.12}\PY{o}{/}\PY{n}{vvmax}\PY{p}{)}
        \PY{n}{ax1}\PY{o}{.}\PY{n}{plot}\PY{p}{(}\PY{n}{x}\PY{p}{,}\PY{n}{y}\PY{p}{)}
        \PY{n}{y}\PY{o}{=}\PY{o}{.}\PY{l+m+mi}{36}\PY{o}{*}\PY{p}{(}\PY{n}{x}\PY{o}{\PYZhy{}}\PY{o}{.}\PY{l+m+mf}{0e\PYZhy{}4}\PY{p}{)}
        \PY{n}{ax1}\PY{o}{.}\PY{n}{plot}\PY{p}{(}\PY{n}{x}\PY{p}{,}\PY{n}{y}\PY{p}{,}\PY{n}{linestyle}\PY{o}{=}\PY{l+s+s1}{\PYZsq{}}\PY{l+s+s1}{dotted}\PY{l+s+s1}{\PYZsq{}}\PY{p}{)}\PY{p}{;}
\end{Verbatim}

    \begin{center}
    \adjustimage{max size={0.9\linewidth}{0.9\paperheight}}{scaling_lab_events_files/scaling_lab_events_10_0.pdf}
    \end{center}
    { \hspace*{\fill} \\}
    
    Fig. 4. Total slip for each rupture event as a function of rise time
(red dots). The theoretical fit (blue curve) is shown assuming the
classic scaling relation \[U=C\ \frac{\sigma\Delta\mu}{\mu'}\ \Gamma\]
where \(\Gamma\) is the length of the rupture, \(\mu'\) is the shear
stiffness, \(\sigma\) is the normal stress and \(C\) is a geometrical
constant of the order of 1. Furthermore, we may use
\[\Gamma\approx T\ V_r\] where \(T\) is rise time and \(V_r\) is rupture
velocity. According to approximate relation of stress drop to maximum
slip velocity as discussed in Figure 3, we have: \[
\Delta\mu=\mu_0(1-V_w/V_{max})
\] And according to the fit of Figure 5, we may replace
\(V_{max}=0.12+(\frac{T}{8\ 10^{-4}})^2\). As a result we obtain the
relation: \[
\begin{split}
U=\frac{C\ \sigma\ \mu_0\ V_r}{\mu'} \left(1-\frac{V_w}{0.12+(\frac{T}{8\ 10^{-4}})^2}\right)\ T
\\
=0.36\ \left(1-\frac{V_w}{0.12+(\frac{T}{8\ 10^{-4}})^2}\right)\ T
\end{split}
\] where we have used the indicative values, compatibly with the
experimental conditions, of \(\sigma\approx 70\mathrm{ MPa}\),
\(V_r\approx 1000\mathrm{ (m/s)}\), \(\mu'=50\ \mathrm{ GPa}\),
\(\mu_0=0.5\), and set the constant \(C=0.6\) to obtain the fit of the
experimental points. The asymptotic value \(U=0.36\ T\) at large \(T\)
is shown as a dotted line. The linear asymptote corresponds to the
maximum possible friction drop (possibly close to total drop or
\(\Delta\mu\approx\mu_0=0.5\)) which is achieved at large \(V_{max}\)
(and large \(T\)), and whereby self-similar scaling is retrieved.

    \section{Export to LaTeX (without code
cells)}\label{export-to-latex-without-code-cells}

    \begin{Verbatim}[commandchars=\\\{\},fontsize=\scriptsize]
{\color{incolor}In [{\color{incolor}74}]:} \PY{k+kn}{import} \PY{n+nn}{os}
         \PY{n}{os}\PY{o}{.}\PY{n}{system}\PY{p}{(}\PY{l+s+s1}{\PYZsq{}}\PY{l+s+s1}{mkdir build}\PY{l+s+s1}{\PYZsq{}}\PY{p}{)}\PY{p}{;}
         \PY{n}{os}\PY{o}{.}\PY{n}{system}\PY{p}{(}\PY{l+s+s1}{\PYZsq{}}\PY{l+s+s1}{jupyter nbconvert \PYZhy{}\PYZhy{}to=latex \PYZhy{}\PYZhy{}template=latex\PYZus{}nocode.tplx scaling\PYZus{}lab\PYZus{}events.ipynb }\PY{l+s+s1}{\PYZsq{}}\PY{p}{)}
         \PY{c+c1}{\PYZsh{}the latex\PYZus{}nocode.tplx will eliminate the code cells \PYZhy{} the file latex\PYZus{}article.tplx is also needed}
\end{Verbatim}

\begin{Verbatim}[commandchars=\\\{\}]
{\color{outcolor}Out[{\color{outcolor}74}]:} 0
\end{Verbatim}
            
    \section{compile pdflatex and visualise the
result:}\label{compile-pdflatex-and-visualise-the-result}

    \begin{Verbatim}[commandchars=\\\{\},fontsize=\scriptsize]
{\color{incolor}In [{\color{incolor}108}]:} \PY{n}{os}\PY{o}{.}\PY{n}{system}\PY{p}{(}\PY{l+s+s1}{\PYZsq{}}\PY{l+s+s1}{pdflatex \PYZhy{}output\PYZhy{}directory=build scaling\PYZus{}lab\PYZus{}events}\PY{l+s+s1}{\PYZsq{}}\PY{p}{)}
          \PY{n}{os}\PY{o}{.}\PY{n}{system}\PY{p}{(}\PY{l+s+s1}{\PYZsq{}}\PY{l+s+s1}{open build/scaling\PYZus{}lab\PYZus{}events.pdf}\PY{l+s+s1}{\PYZsq{}}\PY{p}{)}
\end{Verbatim}

\begin{Verbatim}[commandchars=\\\{\}]
{\color{outcolor}Out[{\color{outcolor}108}]:} 0
\end{Verbatim}
            
    \section{Export to LaTeX (including code
cells)}\label{export-to-latex-including-code-cells}

    \begin{Verbatim}[commandchars=\\\{\},fontsize=\scriptsize]
{\color{incolor}In [{\color{incolor}107}]:} \PY{k+kn}{import} \PY{n+nn}{os}
          \PY{n}{os}\PY{o}{.}\PY{n}{system}\PY{p}{(}\PY{l+s+s1}{\PYZsq{}}\PY{l+s+s1}{mkdir build}\PY{l+s+s1}{\PYZsq{}}\PY{p}{)}
          \PY{n}{os}\PY{o}{.}\PY{n}{system}\PY{p}{(}\PY{l+s+s1}{\PYZsq{}}\PY{l+s+s1}{jupyter nbconvert \PYZhy{}\PYZhy{}to=latex \PYZhy{}\PYZhy{}template=article\PYZus{}plus.tplx scaling\PYZus{}lab\PYZus{}events.ipynb }\PY{l+s+s1}{\PYZsq{}}\PY{p}{)}
          \PY{c+c1}{\PYZsh{}the latex\PYZus{}nocode.tplx will eleiminate the code cells \PYZhy{} the file latex\PYZus{}article.tplx is also needed}
\end{Verbatim}

\begin{Verbatim}[commandchars=\\\{\}]
{\color{outcolor}Out[{\color{outcolor}107}]:} 0
\end{Verbatim}
            

    % Add a bibliography block to the postdoc
    
    
    
    \end{document}
